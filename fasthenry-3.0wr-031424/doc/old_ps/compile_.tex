%
% Description of output compile flag effects and compile procedure
%
\section{Compiling FastHenry}
\label{comfas}

%\noindent{\sc Check particulars w/Miguel.}

A tar file containing the source files for {\tt fasthenry} and this
guide may be obtained on tape by sending a written request to
\begin{quote}
Prof.\ Jacob White\\
Massachusetts Institute of Technology\\
Department of Electrical Engineering and Computer Science\\
Room 36-880\\
Cambridge, MA 02139 U.S.A.
\end{quote}
This address may also be used for  general correspondence regarding 
{\tt fasthenry}, although electronic mail may be sent to
{\tt fasthenry-bug@rle-vlsi.mit.edu}, 
for bug reports, and to {\tt fasthenry@rle-vlsi.mit.edu}, for
questions or comments, if it is more convenient.  Comments on
FastHenry are encouraged as well as comments on unclear portions of
this manual.

The tar file has the form
\begin{quote}
\begin{verbatim}
fasthenry-3.0-9Oct96.tar.Z
\end{verbatim}
\end{quote}
and yields a one level directory when untarred with the commands
\begin{quote}
\begin{verbatim}
uncompress fasthenry-3.0-9Oct96.tar.Z
tar xvf fasthenry-3.0-9Oct96.tar
\end{verbatim}
\end{quote}
It will create a directory tree underneath {\tt fasthenry-3.0} which 
contains all the C source files underneath {\tt src/}, this manual in
{\tt doc/}, and the example files in {\tt examples/}.

The tar files may
also be obtained via anonymous ftp to {\tt rle-vlsi.mit.edu}.  Use
username {\tt anonymous} with your email address as the password.  The
FastHenry tar files are contained in directory {\tt pub/fasthenry}.
This directory also contains various publications on multiplole
accelerated inductance extraction.  The most complete description is 
the compressed postscript file {\tt ms\_thesis.ps.Z}:

\begin{quote}
\noindent
Mattan Kamon, {\em Efficient Techniques for Inductance Extraction of
Complex 3-D Geometries}, M.S. thesis, Massachusetts Institute of
Technology, Cambridge, MA., February 1994.
\end{quote}

\noindent {\tt mtt.ps.Z} is a preprint of 

\begin{quote}
\noindent
M. Kamon, M.J. Tsuk, and J. White, ``FASTHENRY: A
Multipole-Accelerated 3-D Inductance Extraction Program'', {\em IEEE
Transactions on Microwave Theory and Techniques}, Vol.~42, 
September 1994.
\end{quote}

\noindent See the {\tt README} file at the ftp site for update information on
releases, manuals, publications.

\subsection{Compilation Procedure}

FastHenry is compiled by changing to the {\tt fasthenry-3.0} directory, 
and typing
\begin{quote}
\begin{verbatim}
make all
\end{verbatim}
\end{quote}
to create the executables {\tt fasthenry} and {\tt zbuf} in the {\tt
bin/} directory.
%This will use the file {\tt Makefile} to make {\tt fasthenry}. 

If you are compiling on a DEC 5000, DEC 3000 (Alpha), SGI, or a system
running System V (HP and Suns running Solaris, perhaps), 
other flags are required for compilation.  In this
case, before compilation as instructed above, give the command:
\begin{quote}
\begin{verbatim}
config <name>
\end{verbatim}
\end{quote}
where {\tt <name>} is one of {\tt dec}, {\tt alpha}, or {\tt sgi}, or
{\tt solaris} for
compilation on a DEC 5000, DEC 3000 (Alpha), and Silicon Graphics
workstation, respectively.  Additional steps for compiling on an SGI
are given in the file {\tt README.sgi} in the {\tt fasthenry-3.0}
directory. {\tt config sysV} is also provided for
compilation on HP-UX, and other machines running System V,
but has not been thoroughly tested.  For more details on compilation,
see the {\tt fasthenry-3.0/README} file.

\subsection{Producing this Guide}

This guide is available in postscript as three separate files: {\tt
manual\_001.ps, manual\_002.ps}, and {\tt manual\_003.ps} in
the {\tt doc/} directory.  {\tt manual\_001.ps} contains up through
page 13, {\tt manual\_002.ps} is pages 14--31, and {\tt manual\_003.ps}
is pages 32--36.  Note that these files contain many detailed
postscript images and may take significant time to print.  The
\LaTeX\ version of the manual without the figures is also available.

Also, the nonuniform plane discretization manual is in the files {\tt
  nonuniform\_manual\_1.ps} and {\tt nonuniform\_manual\_2.ps }
